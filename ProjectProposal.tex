\documentclass{article}
\usepackage[utf8]{inputenc}
\title{
	\vskip -3cm
	Project Proposal: Hackcell \\
	\vskip 0.5cm
	\large{A spreadsheet library in Haskell}
}
\author{Joris Burger, Lars van den Haak \& Ivo Gabe de Wolf}
\begin{document}
	\maketitle
	\section{Domain Introduction}
	The aim of the project is to provide a library for spreadsheet calculations in Haskell. A spreadsheet is an interactive program where users can perform calculations. It consists of cells, where values or formulas are stored. The cells can be displayed, and will show either the value it contains or the result of the formula it contains. In formulas references to other cells can be made and as soon as one of the referenced cell is changed, the result of the formula should be updated. Also a range of cells can be provided as argument of a function, for example to calculate the sum or average of a range of cells. 
	
	A common spreadsheet application is Excel, where the cells are stored in a two dimensional sheet and a file can contain multiple sheets. Cells can refer to other cells in the same or other sheets. Therefor Excel is three dimensional. Most of the calculations are made on numbers or text. 
	
	\section{Problem Description}
	
	\section{Schedule}
\end{document}