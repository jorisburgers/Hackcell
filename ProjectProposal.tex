\documentclass{article}
\usepackage[utf8]{inputenc}
\title{
	\vskip -3cm
	Project Proposal: Hackcell \\
	\vskip 0.5cm
	\large{A spreadsheet library in Haskell}
}
\author{Joris Burgers - 5545358\\ Lars van den Haak - 3867803\\ Ivo Gabe de Wolff - 4279433}
\begin{document}
	\maketitle
	\section{Domain Introduction}
	The aim of the project is to provide a library for spreadsheet calculations in Haskell. A spreadsheet is an interactive program where users can perform calculations. It consists of cells, where values or formulas are stored. The cells can be displayed, and will show either the value it contains or the result of the formula it contains. In formulas references to other cells can be made and as soon as one of the referenced cell is changed, the result of the formula should be updated. Also a range of cells can be provided as argument of a function, for example to calculate the sum or average of a range of cells. 
	
	A common spreadsheet application is Excel, where the cells are stored in a two dimensional sheet and a file can contain multiple sheets. Cells can refer to other cells in the same or other sheets. Therefor Excel is three dimensional. Most of the calculations are made on numbers or text. 
	
	\section{Problem Description}
	The problem that Hackcell will solve is the calculation of spreadsheets. This means that, given a spreadsheet with combined values and expressions, Hackcell will provide a spreadsheet where all expressions are replaced with their appropriate value. This spreadsheet can be of arbitrary size and dimensions. There is a number of problems that Hackcell will solve.
\begin{itemize}
	\item The ability to allow arbitrary data types in the same spreadsheet. Users can use their own types and are not limited to a number of predefined types. 
	\item Remove the limitation of the number of dimensions. Most spreadsheet programs give the user a limited amount of dimensions to work with. We want to remove this restriction and let a user define a spreadsheet of arbitrary size and dimensions.
	\item Detection of cyclic references. In a spreadsheet, cells can reference other cells and this can cause a cycle. This cycle will be detected, preventing an endless loop.
	\item The possibility to allow users to update specific cells without the need to recompute the entire spreadsheet. The only cells that will be updated are those cells that depend on the changed cell.
	\item The only values that are computed are the values that need to be computed. This can be done using the laziness that is already present in Haskell. 
\end{itemize}	
	
	\subsection{Proposed solution}
	We propose to create a library with a generic core and an extension, that will replicate the behavior of Excel. We try to make the core generic, such that arbitrary types can be used for both the keys and values in the spreadsheet. The core will provide an interface to create a spreadsheet and update cells of an existing spreadsheet. It can compute the values of cells in the spreadsheet.
	
	The core provides an interface (DSL), probably monadic, to write functions like \texttt{SUM} and \texttt{AVERAGE} in Excel. This interface will be able to get the value of referenced cells and mark them as dependencies. It will also have a function to denote that the computation of a cell failed, for instance because of a division by zero.
	
	This abstraction makes it easy to construct a dependency graph. We use this graph to recalculate only a minimal number of cells when one cell changes.
	
	We will use the core to build an extension that replicates the functionality of Excel. This will have the type of the key and the values fixed. Note that cells can still have different types, but they are all mapped to a unitype like \texttt{data Prim = PInt Int | PBool Bool | PDouble Double}. Depending on time, we can add some more complex functions and maybe add some SQL-like query functions.
	
	\section{Schedule}
\end{document}