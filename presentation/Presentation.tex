\documentclass{beamer}
\usepackage{hyperref}
\usepackage{minted}
\usepackage{beamerthemeuucs}
\title{Hackcell}
\subtitle{A spreadsheet library in Haskell}
\author{Joris Burgers, Lars van den Haak and Ivo Gabe de Wolff}
\date{April 5, 2018}

\begin{document}
	\frame{\titlepage}
  \begin{frame}
	\begin{center}
		\huge{Demonstration}
	\end{center}
  \end{frame}

  \begin{frame}
  		\frametitle{Wrapper}
		\begin{itemize}  		
  			\item The wrapper is a shell around the abstract \textit{Core}
  			\item A programmer can construct their own wrapper or use an existing one
  			\item The wrapper provides the concrete fields and values of the spreadsheet.
  		\end{itemize}
  \end{frame}
  \begin{frame}
  		\frametitle{Wrapper}
  		For a wrapper, the programmer needs a few things.
  		\begin{itemize}
  			\item \texttt{Field}: A way to index every cell.
  			\item \texttt{Value}: The content of a cell, can be any type.
  			\item \texttt{Error}: The errors that can be thrown.
  			\item \texttt{Applications}: The commands that can be performed on the values.
  		\end{itemize}
  \end{frame}
  \begin{frame}[fragile]
  		\frametitle{Wrapper - Field}
  		\begin{minted}{haskell}
data Field = Field Int
            deriving (Ord)
  		\end{minted}
  		\pause
		\begin{minted}{haskell}
data Field = Field (String, Int, Int)
            deriving (Ord)
  		\end{minted}  		
  		\pause
  		\begin{minted}{haskell}
data Field = Field (String, Int, Int)
	   | CurrentTime
           deriving (Ord)
        \end{minted} 
  \end{frame}
\begin{frame}[fragile]
  		\frametitle{Wrapper - Value}
  		\begin{minted}{haskell}
data Value  = ValInt Int
            | ValDouble Double
            | ValBool Bool
            | ValString String
            | ValTree Tree
  		\end{minted}
  \end{frame}
  \begin{frame}[fragile]
  		\frametitle{Wrapper - Error}
		  		\begin{minted}{haskell}
data NumberError    = RecursionError [Field]
                    | UnknownFieldError String
                    | DivideByZeroError String
                    | ErrorUnexpectedValue
                    | ErrorUnexpectedRange
                    | ErrorInvalidType
                    deriving (Show)
  		\end{minted}
  \end{frame}
    \begin{frame}[fragile]
  		\frametitle{Wrapper - Error}
		  		\begin{minted}{haskell}

class HackcellError t field | t -> field where
  errorUnknownField :: field -> t
  errorRecursion :: [field] -> t
  errorExpectedValueGotRange :: t
  errorExpectedRangeGotValue :: t
  		\end{minted}
  		The function \texttt{tError} can be used to throw an error for the current cell.
  \end{frame}
	    \begin{frame}[fragile]
  		\frametitle{Wrapper - Application}
  		Handler takes a command and a list of parameters.
  		\\\\
  		Example commands:
  		\begin{minted}{haskell}
data Commands = Plus
              | Minus
              | Times
              | Divide        
  		\end{minted}
  \end{frame}  
  \begin{frame}[fragile]
  \frametitle{Wrapper - Application}
  		Handler takes a command and a list of parameters.
  		\\
  		Example \texttt{PLUS}:
  		\begin{minted}{haskell}
apply Plus [px, py] = do
		ValueInt x <- expectValue px 
		ValueInt y <- expectValue py
		return (ValueInt (x + y))
  		\end{minted}
  		\begin{itemize}
  		\item \texttt{expectValue} is used for values
  		\item \texttt{expectRange} is used to get lower bound and upper bound on ranges.
  		\end{itemize}
  \end{frame}  

\begin{frame}
\frametitle{Core}
\begin{itemize}
	\item Generic
	\item Simple expression language
	\item Memoization
	\item Dependency tracking
	\item Monadic interface for method definitions
\end{itemize}
\end{frame}

\begin{frame}[fragile]
\frametitle{Type arguments}
\begin{minted}{haskell}
newtype Spreadsheet field value err app =
  Spreadsheet { unSpreadsheet :: M.Map field
    (Expression field value err app) }
\end{minted}
\end{frame}

\begin{frame}[fragile]
\frametitle{Type classes}
\begin{minted}{haskell}
class Ord field

class Apply field value error app where
  apply :: app
        -> [Argument field value error app]
        -> Eval field value error app value

\end{minted}
\end{frame}

\begin{frame}[fragile]
\frametitle{Expression language}
\begin{minted}{haskell}
data Expression field value error app
  = ExprField field
  | ExprLit value
  | ExprApp app [Parameter field value error app]

data Parameter field value error app
  = PExpr (Expression field value error app)
  | PRange field field
\end{minted}
\end{frame}

\begin{frame}[fragile]
\frametitle{Memoization \& dependency tracking}
\begin{itemize}
	\item Store the value
	\item Store a list of fields that depend on this field
	\item Dependency graph
\end{itemize}

\begin{minted}{haskell}
M.Map field (
  Expression field value error app,
  Maybe (FieldResult field value error))

data FieldResult field value error = FieldResult
  (Either error value)
  [field]
\end{minted}
\end{frame}

\begin{frame}
\frametitle{Updating a field}
\begin{itemize}
	\item Remove old expression \& result
	\item Recursively invalidate the dependents
	\item When requested, evaluate the field
\end{itemize}
\end{frame}

\begin{frame}[fragile]
\frametitle{Eval monad}

\texttt{EvalState}
\begin{itemize}
	\item The current field, used for dependency tracking
	\item A stack of fields to detect circular references
	\item The state of the spreadsheet
\end{itemize}

\begin{minted}{haskell}
newtype Eval field value error app a = Eval
  (ExceptT error
    (State (EvalState field value error app))
    a)
\end{minted}
\end{frame}

\frame{\titlepage}

\end{document}